\documentclass[journal,12pt,twocolumn]{IEEEtran}

\usepackage{setspace}
\usepackage{gensymb}

\singlespacing


\usepackage[cmex10]{amsmath}

\usepackage{amsthm}
\usepackage{ulem}

\usepackage{mathrsfs}
\usepackage{txfonts}
\usepackage{stfloats}
\usepackage{bm}
\usepackage{cite}
\usepackage{cases}
\usepackage{subfig}

\usepackage{longtable}
\usepackage{multirow}

\usepackage{enumitem}
\usepackage{mathtools}
\usepackage{steinmetz}
\usepackage{tikz}
\usepackage{circuitikz}
\usepackage{verbatim}
\usepackage{tfrupee}
\usepackage[breaklinks=true]{hyperref}
\usepackage{graphicx}
\usepackage{tkz-euclide}

\usetikzlibrary{calc,math}
\usepackage{listings}
    \usepackage{color}                                            %%
    \usepackage{array}                                            %%
    \usepackage{longtable}                                        %%
    \usepackage{calc}                                             %%
    \usepackage{multirow}                                         %%
    \usepackage{hhline}                                           %%
    \usepackage{ifthen}                                           %%
    \usepackage{lscape}     
\usepackage{multicol}
\usepackage{chngcntr}

\DeclareMathOperator*{\Res}{Res}

\renewcommand\thesection{\arabic{section}}
\renewcommand\thesubsection{\thesection.\arabic{subsection}}
\renewcommand\thesubsubsection{\thesubsection.\arabic{subsubsection}}

\renewcommand\thesectiondis{\arabic{section}}
\renewcommand\thesubsectiondis{\thesectiondis.\arabic{subsection}}
\renewcommand\thesubsubsectiondis{\thesubsectiondis.\arabic{subsubsection}}


\hyphenation{op-tical net-works semi-conduc-tor}
\def\inputGnumericTable{}                                 %%

\lstset{
%language=C,
frame=single, 
breaklines=true,
columns=fullflexible
}
\begin{document}


\newtheorem{theorem}{Theorem}[section]
\newtheorem{problem}{Problem}
\newtheorem{proposition}{Proposition}[section]
\newtheorem{lemma}{Lemma}[section]
\newtheorem{corollary}[theorem]{Corollary}
\newtheorem{example}{Example}[section]
\newtheorem{definition}[problem]{Definition}

\newcommand{\BEQA}{\begin{eqnarray}}
\newcommand{\EEQA}{\end{eqnarray}}
\newcommand{\define}{\stackrel{\triangle}{=}}
\bibliographystyle{IEEEtran}
\providecommand{\mbf}{\mathbf}
\providecommand{\pr}[1]{\ensuremath{\Pr\left(#1\right)}}
\providecommand{\qfunc}[1]{\ensuremath{Q\left(#1\right)}}
\providecommand{\sbrak}[1]{\ensuremath{{}\left[#1\right]}}
\providecommand{\lsbrak}[1]{\ensuremath{{}\left[#1\right.}}
\providecommand{\rsbrak}[1]{\ensuremath{{}\left.#1\right]}}
\providecommand{\brak}[1]{\ensuremath{\left(#1\right)}}
\providecommand{\lbrak}[1]{\ensuremath{\left(#1\right.}}
\providecommand{\rbrak}[1]{\ensuremath{\left.#1\right)}}
\providecommand{\cbrak}[1]{\ensuremath{\left\{#1\right\}}}
\providecommand{\lcbrak}[1]{\ensuremath{\left\{#1\right.}}
\providecommand{\rcbrak}[1]{\ensuremath{\left.#1\right\}}}
\theoremstyle{remark}
\newtheorem{rem}{Remark}
\newcommand{\sgn}{\mathop{\mathrm{sgn}}}
\providecommand{\abs}[1]{\left\vert#1\right\vert}
\providecommand{\res}[1]{\Res\displaylimits_{#1}} 
\providecommand{\norm}[1]{\left\lVert#1\right\rVert}
%\providecommand{\norm}[1]{\lVert#1\rVert}
\providecommand{\mtx}[1]{\mathbf{#1}}
\providecommand{\mean}[1]{E\left[ #1 \right]}
\providecommand{\fourier}{\overset{\mathcal{F}}{ \rightleftharpoons}}
%\providecommand{\hilbert}{\overset{\mathcal{H}}{ \rightleftharpoons}}
\providecommand{\system}{\overset{\mathcal{H}}{ \longleftrightarrow}}
	%\newcommand{\solution}[2]{\textbf{Solution:}{#1}}
\newcommand{\solution}{\noindent \textbf{Solution: }}
\newcommand{\cosec}{\,\text{cosec}\,}
\providecommand{\dec}[2]{\ensuremath{\overset{#1}{\underset{#2}{\gtrless}}}}
\newcommand{\myvec}[1]{\ensuremath{\begin{pmatrix}#1\end{pmatrix}}}
\newcommand{\mydet}[1]{\ensuremath{\begin{vmatrix}#1\end{vmatrix}}}
\numberwithin{equation}{subsection}
\makeatletter
\@addtoreset{figure}{problem}
\makeatother
\let\StandardTheFigure\thefigure
\let\vec\mathbf
\renewcommand{\thefigure}{\theproblem}
\def\putbox#1#2#3{\makebox[0in][l]{\makebox[#1][l]{}\raisebox{\baselineskip}[0in][0in]{\raisebox{#2}[0in][0in]{#3}}}}
     \def\rightbox#1{\makebox[0in][r]{#1}}
     \def\centbox#1{\makebox[0in]{#1}}
     \def\topbox#1{\raisebox{-\baselineskip}[0in][0in]{#1}}
     \def\midbox#1{\raisebox{-0.5\baselineskip}[0in][0in]{#1}}
\vspace{3cm}
\title{Assignment-15}
\author{Ankur Aditya - EE20RESCH11010}
\maketitle
\newpage
\bigskip
\renewcommand{\thefigure}{\theenumi}
\renewcommand{\thetable}{\theenumi}

\begin{abstract}
This document contains the problem related to Linear Transformations (Hoffman:- Page-111,Q-1a) 
\end{abstract}
Download the latex-file from 
\begin{lstlisting}
https://github.com/ankuraditya13/EE5609-Assignment15
\end{lstlisting}

\section{Problem}
Let n be a positive integer and $\vec{F}$ a field. let $\vec{W}$ be the set of all vectors $\myvec{\vec{x_1},&\vec{x_2},\cdots ,\vec{x_n}} \in \vec{F}^n$ such that $\vec{x_1}+\vec{x_2}+\vec{x_3}+\cdots \vec{x_n} = 0$. Prove that $\vec{W}^0$ consists of all linear functional f of the form
\begin{align}
f\myvec{\vec{x_1},&\vec{x_2}&,\cdots &,\vec{x_n}} = c\sum_{j=1}^{n}\vec{x_j}
\end{align}
\section{Definitions}
\subsection{Definition 1}
If the vector space $\vec{V}$ is finite dimensional (say dimension =n), the dimension of null-space $\vec{N}_f$ by rank nullity theorem is given by,
\begin{align}
\abs{\vec{N}_f}  = \abs{\vec{V}} - 1 = n - 1
\end{align}  
\subsection{Definition 2}
If $\vec{V}$ is a vector space over the field $\vec{F}$ and $\vec{S}$ is a subset of $\vec{V}$, the annihilator of $\vec{S}$ is the set $\vec{S}^0$ of linear functional f on $\vec{V}$ such that $f(\alpha) = 0$ for every $\alpha$ in $\vec{S}$. 
\section{Solution}
Let h be the functional,
\begin{align}
h\myvec{\vec{x_1},&\vec{x_2},\cdots ,\vec{x_n}} = \vec{x_1}+\vec{x_2}+\cdots \vec{x_n}\\
\mbox{Let, } \vec{X} = \myvec{\vec{x_1}\\\vec{x_2}\\\vdots\\\vec{x_n}}
\end{align}
\begin{align}
\implies h\brak{\vec{X}^T} = 0
\end{align}
Then $\vec{W}$ is in null-space of h. Hence by definition-1, the dimension of $\vec{W}$ is,
\begin{align}
\abs{W} = n - 1
\label{dim}
\end{align}
Now let, 
\begin{align}
a_j = \epsilon_1-\epsilon_{i+1}, \mbox{ for } i = (1,\cdots ,n-1)
\label{aj} 
\end{align}
Hence clearly $\{a_1, a_2, \cdots, a_{n-1}\}$ are linearly independent. Hence from \eqref{dim} and above statement we can conclude that $\{a_1, a_2, \cdots, a_{n-1}\}$ are all in $\vec{W}$ so they must form basis for $\vec{W}$. Now, it is given that f is linear functional hence,
\begin{align}
f\brak{\vec{X}^T} = \sum_{j=1}^{n}c_j\vec{x_j}
\end{align}
\begin{align}
\implies f\brak{\vec{X}^T} = C^T\vec{X}
\end{align}
Where,
\begin{align}
C = \myvec{c_1\\\vdots\\c_n}
\end{align}
Now $f \in \vec{W}^0$ from definition-2 is given as,
\begin{align}
f(a_1) = f(a_2) = \cdots = f(a_n) = 0 \mbox{ for every } a_j \in \vec{W}
\end{align}
\begin{align}
\implies c_1 - c_i = 0 \forall i= 2\cdots n\\
\implies c_i = c_1 = c \forall i
\end{align}
Hence,
\begin{align}
f\myvec{\vec{X}^T} = c\vec{x_1} + c\vec{x_2} + \cdots + c\vec{x_n}\\
\end{align}
\begin{align}
\implies f\myvec{\vec{x_1},&\vec{x_2}&,\cdots &,\vec{x_n}} = c\sum_{j=1}^{n}\vec{x_j}
\end{align}
\centerline{\uline{\bfseries Hence, Proved}}

\end{document}